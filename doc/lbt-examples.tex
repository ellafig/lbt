\documentclass[a4paper,oneside,11pt,article]{memoir}

\usepackage{lbt}
\usepackage[dvipsnames]{xcolor}
\usepackage{newpxtext}
\usepackage{tcolorbox}
\usepackage{caption}
  \captionsetup{labelfont={small,bf,color=blue4},textfont={small,color=blue4},labelsep=quad,margin=10pt}
\usepackage[hidelinks]{hyperref}
\usepackage{cleveref}

\setsecnumdepth{subsubsection}
\renewcommand{\thefootnote}{\textcolor{blue4}{\arabic{footnote}}}

\lbtLogChannels{all}
\lbtTemplateDirectory{PWD/templates}
\lbtGlobalOptions{vector.format = bold}

\lbtDefineLatexMacro{integral=lbt.Math:integral}
\lbtDefineLatexMacro{V=lbt.Math:vector}
\lbtDefineLatexMacro{Vijk=lbt.Math:vectorijk}
\lbtDefineLatexMacro{sm=lbt.Math:simplemath}
\lbtDefineLatexMacro{smallnote=lbt.WS0:smallnote}
\lbtDefineLatexMacro{mathlistand=lbt.Math:mathlistand}   % TODO: add to other tex files
\lbtDefineLatexMacro{mathlistor=lbt.Math:mathlistor}     % TODO: add to other tex files
\lbtDefineLatexMacro{mathlist=lbt.Math:mathlist}   % TODO: add to other tex files
\lbtDefineLatexMacro{mathlistdots=lbt.Math:mathlistdots}   % TODO: add to other tex files
\lbtDefineLatexMacro{mathsum=lbt.Math:mathsum}     % TODO: add to other tex files
\fvset{formatcom=\obeyspaces}
\hfuzz=5pt

% \setcounter{tocdepth}{2}

% ----------------------------------------------------------------------

\begin{document}

\lbtDraftModeOff{}

{\Huge LBT examples}

\settocdepth{subsubsection}
\tableofcontents
\clearpage

\newcommand{\package}[1]{{\color{NavyBlue}\textsf{#1}}}
\newcommand{\code}[1]{{\color{NavyBlue}\texttt{#1}}}
\newcommand{\boldcode}[1]{{\bfseries\color{NavyBlue}\texttt{#1}}}
\newcommand{\lbtlogo}{\textsc{lbt}}

\begin{lbt}
  @META
    TEMPLATE   lbt.Doc.Chapter
    TITLE      Demonstrations of core LBT features
    LABEL      ch-demonstration
    SOURCES    LbtDoc

  +BODY
    SECTION Basics

    SUBSECTION Verbatim text
    TEXT The command \code{VERBATIM} opens a \code{Verbatim} environment from the \package{fancyvrb} package. Note the use of the \lbtlogo{} primitive \boldcode{.v} to provide verbatim text to the command.
    LBTEXAMPLE :: (substitute) /ENDV/>>/ :: .v <<
      VERBATIM .v <<
        10 PRINT "Hello"
        20 GOTO 10
      ENDV
    >>
\end{lbt}

\begin{lbt}
  @META
    TEMPLATE   lbt.Doc.Section
    TITLE      Things outside of lbt.Basic
    LABEL      sec-examples
    SOURCES    LbtDoc, lbt.Questions

  +BODY
    SUBSECTION Questions for worksheets

    TEXT The \package{lbt.Questions} template offers useful commands for typesetting questions, subquestions, and multiple-choice options.

    TEXT Use \code{Q} for a top-level question and \code{QQ} for a question part. Use \code{QQ*} for question parts laid out horizontally.

    LBTEXAMPLE .o vertical :: .v <<
      Q Name three different kinds of clouds.

      Q Evaluate the following.
      QQ $3 + 12 / 4$
      QQ $(3 + 12) / 4$

      Q How many vowels appear in each word?
      QQ* [ncols=3]
      :: appear :: Augustine :: crimsom :: toast :: glyph :: transformer
    >>

    TEXT Use \code{MC} or \code{MC*} to lay out \textbf{multiple-choice options}.

    LBTEXAMPLE .o vertical :: .v <<
      Q Which planet of the solar system has the most moons?
      MC Earth :: Mars :: Jupiter :: Saturn

      Q Which planet of the solar system has the fewest moons?
      MC* [ncols=4] :: Mercury :: Venus :: Uranus :: Neptune
    >>

    SUBSUBSECTION Further things to know

    TEXT \dots

\end{lbt}

\end{document}
